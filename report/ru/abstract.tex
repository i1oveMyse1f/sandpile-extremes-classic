\begin{abstract}
     В моделях самоорганизованной критичности, основанных на модели <<куча песка>>, событиям, которые расположены на <<хвосте>> вероятностного распределения событий по размерам, предшествует определённое затишье. Это свойство позволяет прогнозировать время наступления таких событий, при чём прогноз становится эффективнее с увеличением размера прогнозируемых событий. В этой работе мы оценили прогнозируемость в моделях Манна и Бака-Танга-Визенфельда (БТВ) на квадратной решётке в термодинамическом пределе, когда объём системы стремится к бесконечности. Для обеих моделей реализован алгоритм, прогнозирующий наступление крупных событий после уменьшения активности. Установлено совпадение эффективности прогноза на различных решётках для каждой из моделей после нормировки размера событий степенной функцией длины решётки. Показатели степени равны $2.75$ и $3$ для моделей Манна и БТВ соответственно. Из этого следует, что в модели БТВ, в отличие от модели Манна, прогноз в термодинамическом пределе невозможен, по крайней мере на основе предшествующего затишься.
    \newline
    \newline
    Ссылка на гитхаб с проектом - \url{https://github.com/i1oveMyse1f/sandpile-extremes-classic}.
    \newline
    \newline
    \textbf{\textit{Ключевые слова---}} Самоорганизованные критические системы, Модель Манна, Модель БТВ, Скейлинг.
    \newpage
    The state-of-the-art in the theory of self-organized criticality exposes that a certain quiescence precedes events that are located on the tail of the probability distribution of events with respect to their sizes. The existence of the quiescence allows us to predict the occurrence of these events in advance. In this work, we estimate the predictability of the Bak-Tang-Wiesenfeld (BTW) and Manna models on the square lattice in the thermodynamic limit defined by the tendency of the system volume to infinity. For both models, we define an algorithm that forecasts the occurrence of large events after a fall in activity. The collapse of the algorithm efficiency computed with various lattices is found if the size of events is normalized by a power-law function of the lattice length. The power-law exponents are $2.75$ and $3$ for the Manna and BTW models respectively. This yields that the prediction in thermodynamic limit does not exist in the BTW but not in the Manna model, at least based on the quiescence.
    \newline
    \newline
    Github project link - \url{https://github.com/i1oveMyse1f/sandpile-extremes-classic}.
    \newline
    \newline
    \textbf{\textit{Keywords---}} Self-organized criticality, Manna model, BTW model, Scalability.
\end{abstract}
