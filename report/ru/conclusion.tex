%Абзац 1 (Мы можем предсказывать в обоих моделях in general). 
В этой работе мы привели пример скейлинг-эффективной модели прогнозирования крупных событий. В отличии от предыдущих результатов, наш алгоритм работает одинаково как для модели Манна, так и для модели БТВ, и имеет тесную связь с известными теоретическими исследованиями. Таким образом, мы собрали все прошлые результаты в стройную схему, которая связывает размеры решетки и эффективность прогнозирования. 

% Абзац 2 (Термодинамические параметры) 
Оказалось, что эффективные параметры для скейлинга событий равны $\gamma=2.75$ в модели Манна и $\gamma=3$ в модели БТВ. Это соответствует показателю шкалирования для плотности распределения событий в модели Манна и показателю шкалирования размера максимального события в модели БТВ соответственно. За счёт этого, в модели Манна для достаточно редких событий качество прогноза не зависит от размера решетки $L$. Однако, в модели БТВ степенная часть имеет отличный от $s_{\max}$ скейлинг, равный $L^2$, из-за этого качество прогноза относительно частоты встречаемости событий падает с ростом размера решетки. Как итог, прогнозировать крупные события в модели Манна получается более эффективно, чем в модели БТВ, и эта разница растет с ростом $L$.

%Абзац 3 (про эффект конечного размера и про то что в реальности показатели не равны термодинамическому пределу). 
На практике численные эксперименты показали, что для решеток малого размера эффективными показателями для скейлинга являются чуть меньшие параметры $\gamma \approx 2.67$ в модели Манна и $\gamma \approx 2.93$ в модели БТВ. Это объясняется эффектом конечного размера решетки, наблюдаемом и на графиках плотности распределения размеров событий. Мы предполагаем, что для наиболее эффективного прогноза стоит плавно увеличивать $\gamma$ с ростом $L$, не достигая термодинамического предела.

%Абзац 4 (Про память).
Крупным событиям в модели песчаной кучи предшествует накопление песка, что на физическом языке означает переход системы из критического в надкритическое состояние. Алгоритм прогноза в данной статье связан с существованием этого перехода. Видимо, на больших временах модель песчаной кучи обладает памятью степенного размера об отсутствии крупных событий. Было бы интересно понять, можно ли существенно поднять качество прогноза за счёт перехода от одной переменной принятия решения $y$, учитывающей память, к вектору переменных принятия решения, сделав модель более сложной, например, с помощью нейронных сетей.

