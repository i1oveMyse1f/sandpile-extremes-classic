В XX веке исследователи часто наблюдали степенной закон в разных физических явлениях, например, в сейсмичности~\cite{Burridge1967ModelAT}, солнечной активности~\cite{Dennis1985SolarHX}, распределении богатства~\cite{Levy1997NEWEF}. Однако, долгое время не существовало теории и математической модели для объяснения степенного закона, пока в 1987 году Бак, Танг и Винфельд не предложили теорию самоорганизованной критичности и модель песчаной кучи~\cite{btw} (\textit{англ., sandpile}), как её архитипичный пример. Модификации модели песчаной кучи применимы к моделированию сейсмичности~\cite{Khodaverdian2016,TURCOTTE1999275}, взаимодействию нейронов в мозге~\cite{Bak1996HowNW}, солнечной активности~\cite{Aschwanden2021SelforganizedCI}, естественных языков~\cite{Gromov2020} и других явлений~\cite{bunde,Kalinin2021,podgornik}.

Эволюция в модели песчаной кучи Бака–Танга–Визенфельда (БТВ)~\cite{btw} определяется на квадратной решётке со стороной в $L$ клеток. В каждой клетке находится от $0$ до $3$ песчинок. Каждый момент времени выбирается случайная клетка, в которую добавляется одна песчинка. Если в клетке оказывается $4$ песчинки или больше, то клетка называется нестабильной и происходит обвал: из нестабильной клетки в каждую из соседних клеток перемещается по одной песчинке; если соседней клетки нет, то песчинка падает за пределы решетки. Обвалы происходят, пока существуют нестабильные клетки. Считается, что обвалы происходят моментально, то есть до добавления новой песчинки на решётку. Последовательный процесс обвалов называется событием; размером $i$-го события считается количество обвалов $s_i$, которое произошло до стабилизации кучи после падения $i$-й песчинки.

Манна~\cite{manna} предложил другую реализацию модели песчаной кучи, определив стохастический процесс пересыпания песчинок вместо детерминистического. Измененные правила обвала выглядят следующим образом: из нестабильной клетки $4$ раза равновероятно выбирается случайная соседняя клетка, в которую перемещается одна песчинка. Эта модель так же позволяет наблюдать степенной закон, но c другими параметрами. Позже выяснилось, что пока геометрия решетки квадратная, а правила обвала --- симметричные, других степенных законов, отличных от моделей БТВ и Манна, получить не удаётся~\cite{BenHur1996,Dhar2006}.

За счёт простоты модели песчаной кучи удалось исследовать большое количество её свойств, которые позже наблюдались и в жизни~\cite{Held1990,Jaeger1989}. Однако часто эти свойства можно наблюдать только на больших решетках~\cite{Liu1991}. Поэтому исследователи уделяют отдельное внимание термодинамическому пределу --- свойству модели при размере решетки, стремящемуся к бесконечности. Например, выяснилось, что термодинамический предел для шкалируемости, или скейлинга, плотности событий в модели Манна равен $L^{2.75}$~\cite{manna,Vespignani2000AbsorbingstatePT,Dhar2006}. Это значит, что начиная с какого-то $L$ плотности распределений событий будут совпадать, если сжать их по оси размера событий в $L^{2.75}$ раз. Схожий результат о плотности распределения известен и в модели БТВ: размер максимального события $s_{\max} \propto L^3$~\cite{Garber2009}.

Отдельным важным вопросом в модели песчаной кучи является возможность прогнозировать критические, или крупные, события. Долгие годы считалось, что предсказывать крупные события в данной модели невозможно~\cite{Geller1997,Wyss1997,Milovanov2021}, но в последнее время появилось множество разных подходов. Например, в работе~\cite{deluca} был представлен способ прогнозирования на основе расстояния между событиями одного и того же размера для модели Манна. А в работах ~\cite{Hallerberg2009,Kantz2010} предложили использовать условную вероятность крупного события в зависимости от переменной принятия решений, содержащую информацию о предыдущих событиях для модели БТВ.

Целью нашего исследования является проведение сравнительного анализа прогнозируемости (как свойства системы) в моделях БТВ и Манна. Мы намерены описать скейлинг эффективности прогноза относительно длины решётки и оценить прогнозируемость в термодинамическом пределе для каждой из моделей.
